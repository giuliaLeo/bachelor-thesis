\chapter{Codice sviluppato: completo e commentato}
\label{appendix:code}
\begin{center}
	\lstinputlisting[
		language={NetLogo}, 
		firstline=1, 
		lastline=114,
		caption={Codice NetLogo che, tenendo conto della percezione del rischio, modella la diffusione di un'epidemia su due diversi tipi di rete.}, 
		label={list:NetLogo_risk_perception}, 
		frame={none},
		basicstyle={\small\ttfamily}
	]{other/code/riskperception-turtles-with-comments.nlogo}
\end{center}

%\begin{center}
%	\begin{lstlisting}[autogobble,language={NetLogo},caption={Codice NetLogo che modella la diffusione di un'epidemia su due diversi tipi di rete tenendo conto della percezione del rischio.},label={list:NetLogo_risk_perception}]
%;; qui vengono specificate le variabili in termine delle quali si descrive lo stato degli agenti ;;
%turtles-own [state infected? alert]
%;; procedura di setup, il cui primario obiettivo è quello di ripulire il mondo da precedenti esecuzioni del modello ;;
%;: crea innanzitutto 2 agenti, li fa allontanare di 5 passi e li lega mediante un arco ;;
%;; mediante un'estrazione di un numero casuale fra 0 e 1, genera una certa quantità di infetti ;;
%;; mentre tutti gli altri sono suscettibili ;;
%to setup
%  clear-all
%  no-display
%  set-default-shape turtles "circle"
%  ;; create two turtles (nodes) and space them out
%  create-turtles 2 [
%    fd 5
%  ]
%  ask turtle 0 [ create-link-with turtle 1 ]
%  repeat num-nodes [
%    create-network
%  ]
%  ask turtles [
%    set state 0
%    set infected? false
%  ]
%  ask turtles [
%    if random-float 1 < initial-infected [
%      set state num-states
%      set infected? true
%    ]
%  ]
%  recolor
%  display
%  reset-ticks
%end
%
%;; procedura che regola il processo di diffusione dell'infezione ;;
%to go
%  if (not any? turtles with [color = red] or not any? turtles with [color = green]) [stop] ;; il programma si deve fermare se non ci sono più agenti verdi o rossi
%;; decrementiamo la variabile state, che tiene conto della durata dell'infezione, di un'unità ;;
%  ask turtles [
%    if state > 0 [
%      set state state - 1
%    ]
%  ]
%;; gli agenti contano il numero di vicini infetti e assegnano questo numero alla variabile alert ;;
%  ask turtles with [state > 0] [
%    ask out-link-neighbors with [state = 0] [
%      set alert count out-link-neighbors with [state > 0]
%    ]
%  ]
%;;
%  ask turtles with [state > 0] [
%    ask out-link-neighbors with [state = 0] [
%      if random-float 1 < infectivity * exp(-1 * risk-perception * alert) [
%        set state num-states
%        set infected? true
%      ]
%    ]
%  ]
%  recolor
%  tick
%end
%
%;; metodo grazie al quale gli agenti vengono colorati in base al loro stato di salute ;;
%to recolor
%  ask turtles [
%    ifelse state = 0 [
%      ifelse infected? [
%        set color yellow
%      ][
%        set color green
%      ]
%    ][
%      set color red
%    ]
%  ]
%end
%
%;; metodo che viene chiamato nella procedura to go ;;
%;; seleziona in modo casuale uno degli agenti, al quale viene connesso un nuovo nodo appena generato ;;
%;; il processo viene ripetuto per un numero di volte pari al numero di nodi, impostato tramite slider, così da creare l'intera rete ;;
%to create-network
%  if count turtles > num-nodes [ stop ]
%  let partner one-of turtles
%  if random-float 1 < preferential-attachment [
%    set partner one-of [ both-ends ] of one-of links
%  ]
%  create-turtles 1 [
%    move-to partner
%    fd 1
%    create-link-with partner
%  ]
%  layout
%end
%
%;; procedura con la quale si stabilisce la dimensione degli agenti basandoci sul valore dello slider display-degree ;;
%;; e si opera sull'aspetto della rete, in modo che i nodi legati fra loro non risultino né troppo vicini né troppo lontani ;;
%;; e che non si ammassino in corrispondenza dei bordi ;;
%to layout
%  ask turtles [
%    ifelse display-degree?
%      [ set size sqrt count my-links ]
%      [ set size 1 ]
%  ]
%  layout-spring turtles links 0.2 2.0 0.5
%
%  ask turtles [
%    facexy 0 0
%    fd (distancexy 0 0) / 100
%  ]
%end
%
%;; metodo che riporta il numero di agenti che non si sono mai infettati rispetto al loro numero totale ;;
%to-report fraction-susceptibles
%  report count turtles with [infected? = false] / count turtles
%end
%	\end{lstlisting}
%\end{center}