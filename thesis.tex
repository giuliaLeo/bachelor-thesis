%--------------------------------------------------------------
% thesis.tex 
%--------------------------------------------------------------
% - template for the main file of Informatica@Unifi Thesis 
% - based on Classic Thesis Style Copyright (C) 2008 
%   Andr\'e Miede http://www.miede.de   
%--------------------------------------------------------------

%--------------------DOCUMENT-CLASS----------------------------
\documentclass[openright,titlepage,oneside,fleqn,
	headinclude,12pt,a4paper,footinclude,makeidx]{scrbook}

% sostituire "oneside" con "twoside" per stampa fronte-retro
% aggiungere "hidelinks" per eliminare i box nei link ipertestuali
%--------------------------------------------------------------

%--------------------PACKAGES----------------------------------
\usepackage[italian]{babel} % italian rende i nomi "bibliography" e "index" in italiano
\usepackage[fixlanguage]{babelbib} % per creare la bibliografia in lingua
\usepackage[utf8]{inputenc} 
\usepackage[T1]{fontenc} 
\usepackage[square,numbers]{natbib} 
\usepackage[fleqn]{amsmath}
\usepackage{amssymb}
\usepackage{ellipsis}
\usepackage{listings}
\usepackage{subfig}
\usepackage[format=plain,labelformat=simple,labelsep=colon]{caption}
\usepackage{appendix}
\usepackage{siunitx}
\usepackage{lipsum}
\usepackage{multirow} % per tabelle multirow
\usepackage{comment} % per enviroment comment: \begin{comment}...\end{comment}
\usepackage{dia-classicthesis-ldpkg}

\usepackage{amsthm} %necessaria per theoremstyle
\usepackage{mathabx} %per bigtimes, simbolo di prodotto cartesiano


\usepackage[eulerchapternumbers, linedheaders, subfig, beramono, eulermath, parts, dottedtoc]{classicthesis}
% Se non usiamo dottedtoc per l'indice,
% \usepackage[eulerchapternumbers,linedheaders,subfig,beramono,eulermath, parts]{classicthesis}
% al suo posto usiamo tocstyle:
% \usepackage{tocstyle}
% \usetocstyle{allwithdot}
% PROBLEMA: nome dei capitoli minuscoli
%\renewcommand{\cftchappagefont}{\normalfont}
%\usepackage{tocloft}
%\renewcommand{\cfttoctitlefont}{\hfill\Large\itshape}
%\renewcommand\cfttocfont{\normalfont}

\usepackage{imakeidx}
\makeindex[intoc] % aggiunge al ToC l'indice analitico
% sembra non funzionare con tocstyle
%\makeindex

\usepackage{wrapfig}
\usepackage[italian,noabbrev]{cleveref}
%---------------------------------------------------------------

%--------------------NEWCOMMANDS--------------------------------
\newcommand{\myItalianTitle}{Titolo Italiano\xspace}
\newcommand{\myEnglishTitle}{English Title\xspace}
\newcommand{\myDegree}{Corso di Laurea in Fisica e Astrofisica\xspace}
\newcommand{\myCurriculum}{Generale\xspace}
\newcommand{\myName}{Giulia Leone\xspace}
\newcommand{\myRelatore}{Franco Bagnoli\xspace}
%\newcommand{\myCorelatore}{Gabriele Costa\xspace}
%\newcommand{\myOtherCorelatore}{Letterio Galletta\xspace}
\newcommand{\myFaculty}{Scuola di Scienze Matematiche, Fisiche e Naturali\xspace}
\newcommand{\myUni}{\protect{Università degli Studi di Firenze}\xspace}
\newcommand{\myLocation}{Firenze\xspace}
\newcommand{\myTime}{Anno Accademico 2019-2020\xspace}

% \newcommand{\myVersion}{Version 0.1\xspace} % mi dice already defined
% \renewcommand{\myVersion}{Version 0.1\xspace} % forza, ma tempo potrebbe creare conflitti
\newcommand{\myVersione}{Versione 0.0.1\xspace}


% Diritto d'autore: Attribution-NonCommercial-ShareAlike 4.0 International 
% \newcommand{\mycopyright}{\includegraphics[width=1.5cm]{1-front/copyright/by-nc-sa.png} 
%	\href{https://creativecommons.org/licenses/by-nc-sa/4.0/}{Creative
%		Commons Attribution-NonCommercial-ShareAlike 4.0 International (CC BY-NC-SA 4.0)  }\xspace}

% Diritto d'autore: Attribution-NonCommercial-NoDerivatives 4.0 International
% \newcommand{\mycopyright}{\includegraphics[width=1.5cm]{1-front/copyright/by-nc-nd.png} 
%	\href{https://creativecommons.org/licenses/by-nc-nd/4.0/}{Creative
%		Commons Attribution-NonCommercial-NoDerivatives 4.0 International (CC BY-NC-ND 4.0) }\xspace}

%\newcommand{\mycopyright}{\includegraphics{1-front/copyright/by-nc-sa-compact.png} 
%	\href{https://creativecommons.org/licenses/by-nc-sa/4.0/}{Creative
%		Commons Attribution - NonCommercial - ShareAlike 4.0 International (CC BY-NC-SA 4.0)  }\xspace}
		
\newcommand{\mycopyright}{%\includegraphics[width=0.3cm]{front/copyright/c.png} 
\textcopyright\ \xspace}

%\newcommand{\disps}{dispositivo crittografico }
%\newcommand{\dispp}{dispositivi crittografici }


%\newcommand{\codice}[4]
%{\begin{center}
%\lstinputlisting[language={C},firstline={#1}, lastline={#2},caption={#3},label={#4}]{"D:/UNIFI/Magistrale/Tesi/spectre/primeProbe/spark.c"}
%\end{center}}
	
%\newcommand{\codspark}[3]{
%\begin{center}
%\lstinputlisting[language={C},firstline={#1},lastline={#2},firstnumber={#1},caption={[#3]}]{other/code/spark.c}
%\end{center}}

\newcommand{\arr}[2]{$\text{\texttt{#1}}\left[ \text{\texttt{#2}}\right]$}
\newcommand{\arrdoppio}[3]{$\text{\texttt{#1}}\left[ \text{\texttt{#2}} \left[\text{\texttt{#3}}\right]\right]$}

% \newcommand{\kilobyte}{kB}
% \newcommand{\megabyte}{MB}
% \newcommand{\gigahertz}{GHz}

\newcommand{\tcal}{\mathbb{T}}
\newcommand{\talfa}{\mathbb{T}_{\alpha}}
\newcommand{\tbeta}{\mathbb{T}_{\beta}}

\newcommand{\myfloatalign}{\centering}
% how all the floats will be aligned

\renewcommand{\lstlistingname}{Codice}% Listing -> Codice
\renewcommand{\lstlistlistingname}{Elenco dei Codici}% Listings -> List of Algorithms

\theoremstyle{plain}
\newtheorem{teorema}{Teorema}[chapter]
\theoremstyle{definition}
\newtheorem{definizione}{Definizione}[chapter]
\theoremstyle{definition}
\newtheorem{esempio}{Esempio}[chapter]
%\theoremstyle{plain}
%\newtheorem{corollario}{Corollario}[teorema]
%\theoremstyle{plain}
%\newtheorem{proposizione}{Proposizione}[teorema]
%\theoremstyle{definition}
%\newtheorem{protocollo}{Protocollo}[chapter]
%\newenvironment{sistema}%
%	{\left\lbrace\begin{array}{@{}l@{}}}%
%	{\end{array}\right.}
%--------------------------------------------------------------

%--------------------SETTINGS----------------------------------
\newlength{\abcd} % for ab..z string length calculation
\setlength{\extrarowheight}{3pt} % increase table row height
\captionsetup{format=hang,font=small}

\graphicspath{{other/img/}}

\crefname{listing}{codice}{codici}

% Layout settings
\usepackage{geometry}
\geometry{
	a4paper,
	ignoremp,
	bindingoffset = 1cm, 
	textwidth     = 13.5cm,
	textheight    = 21.5cm,
	lmargin       = 3.5cm, % left margin
	tmargin       = 4cm    % top margin 
}

\lstset{ %
	basicstyle=\small\ttfamily, 		% the size of the fonts that are used for the code
	breaklines=true,                	% sets automatic line breaking
	captionpos=b,                   	% sets the caption-position to bottom
	commentstyle=\color{violet},   		% comment style
	frame=none,							% adds a frame around the code
	frameround=fttt,					% round corner (use f instead t to edge corner)
	keepspaces=true,                	% keeps spaces in text, useful for keeping indentation of code (possibly needs columns=flexible)
	keywordstyle=\color{blue},      	% keyword style
	language=C,               			% the language of the code
	numbers=left,                   	% where to put the line-numbers; possible values are (none, left, right)
	numbersep=5pt,                  	% how far the line-numbers are from the code
	numberstyle=\scriptsize\color{black}, % the style that is used for the line-numbers
	stepnumber=1,                   	% the step between two line-numbers. If it's 1, each line will be numbered
	stringstyle=\color{Brown},  		% string literal style
	tabsize=2                   		% sets default tabsize to 2 spaces
}

%---------------------PARAMETERS-------------------------------
\begin{document}
\frenchspacing
\raggedbottom
%--------------------------------------------------------------

%---------------------FRONTMATTER------------------------------
% A book’s front matter contains such things as the title page, an abstract, a table of contents, a preface, a list of notations, a list of figures, and a list of tables. Some of these front matter pages, such as the title page, are traditionally not numbered.

% The \frontmatter declaration makes the pages numbered in lowercase roman, and makes chapters not numbered, although each chapter’s title appears in the table of contents;
% \pagenumbering{roman} %_scambiare i due seguenti comandi con \frontmatter
% \pagestyle{plain} 
% se usiamo \frontmatter, i comandi soprastanti non sono necessari
\frontmatter 

% Pagina di Titolo
\include{1-front/title-page}

% Dedica
\thispagestyle{empty}
\begin{flushright}
\textit{A <Nome>, \\
frase di dedica.}
\end{flushright}

%\newpage
\bigskip

\thispagestyle{empty}
\begin{flushright}
\textit{"Citazione coltissima"} \textit{Autore, "Titolo", 19xx, rif bibliografia.}
\end{flushright}


\pagestyle{scrheadings}
% the page number jumps to the header (in alto a destra).
 
% \pagenumbering{arabic} 
% usare qui, se si desidera numerazione araba unica delle pagine
% spostare in mainmatter, se vogliamo numerazione romana per front e araba per il resto

% ToC
\tableofcontents

%% Elenco Immagini
%\listoffigures
%
%% Elenco Codici
%\lstlistoflistings
%
%% Elenco Tabelle
%\begingroup
%\let\clearpage\relax
%\let\cleardoublepage\relax
%\let\cleardoublepage\relax
%\vspace*{8ex}
%\listoftables
%\endgroup 

% Citazione
\cleardoublepage
\thispagestyle{empty}
\thispagestyle{empty}
\begin{flushright}
\null\vspace{\stretch {1}}
\emph{"Citazione colta,\\ma colta colta,\\in italiano."\\} 
\null\vspace{5mm}
\emph{"Citazione colta,\\ma colta colta,\\in inglese." \break --- Opera, Autore} \vspace{\stretch{2}}\null
\end{flushright}
\cleardoublepage

%--------------------------------------------------------------

%---------------------MAINMATTER-------------------------------

% The \mainmatter changes the behavior back to the expected version, and resets the page number
\mainmatter
%\pagenumbering{arabic}
% use \cleardoublepage here to avoid problems with pdfbookmark
% il comando sottostante elimina le pagine bianche tra un cap e l'altro
% consiglio l'eliminazione per la stampa fronte retro => cap iniziano sempre sulla pagine destra
% \let\cleardoublepage\clearpage	
% \include{intro} % use \myChapter command instead of \chapter
% \cleardoublepage\myPart{Part I}
% \cleardoublepage\myPart{Part II}

% Capitoli
\addcontentsline{toc}{chapter}{Introduzione}
\chapter*{Introduzione} \markboth{}{Introduzione}
% l'uso di '*' rende il capitolo non numerato ed indicizzato nel toc
% lo aggiungiamo noi al toc tramita la prima istruzione
% con \markboth{<left output>{<right output>} evitiamo che nell'header compaia il titolo del capitolo precedente

%\chapter{Introduzione}
\label{chap:intro}
Il lavoro di tesi è suddiviso in quattro capitoli, il cui contenuto viene qui esposto brevemente:
\begin{itemize}
\item \textit{Capitolo 1}: si descrive cosa significhi modellare un'epidemia, con particolare attenzione a quali siano gli obiettivi che si intendono raggiungere e, al contempo, i limiti che si incontrano nel perseguire questo intento. Dopo aver fatto cenno all'approccio compartimentale, si esplora il modello SIR e si mette in rilievo l'importanza del numero di riproduzione di base $\mathcal{R}_0$; infine, si dedica spazio alla modellizzazione ad agenti, che annovera fra i suoi pregi la capacità di mostrare l'emergere di tutta una serie di comportamenti collettivi che altrimenti non potrebbe venire predetti.
\item \textit{Capitolo 2}: si introducono una serie di concetti teorici legati ai grafi, concetti che saranno utili per meglio descrivere l'andamento di un'epidemia su rete.
\item \textit{Capitolo 3}: si tratta della percezione del rischio e della sua influenza sull'andamento della diffusione dell'infezione; questo elemento viene matematicamente formalizzato andando a modificare la probabilità netta di contrarre la malattia mediante l'aggiunta di due fattori, il primo basato sulla frazione di contatti infetti e l'altro relativo alle misure precauzionali che possono essere messe in atto. Si esplora l'andamento dell'epidemia su due diversi tipi di rete (poissoniana e scale-free) e se ne riportano i risultati sperimentali ottenuti.
\item \textit{Capitolo 4}: facendo riferimento ai concetti introdotti e ai risultati evidenziati nei capitoli precedenti, si spiega come sia stato costruito il modello NetLogo del quale ci si è serviti per condurre una serie di simulazioni; se ne sono poi messi in luce gli esiti e il loro confermare quanto precedentemente ottenuto.
\end{itemize}

\chapter{Titolo del primo capitolo}
\label{chap:cap1}
\chapter{Titolo del secondo capitolo}
\label{chap:cap2}

Le assunzioni secondo cui ogni individuo possa entrare in contatto con chiunque (mixing omogeneo) e il numero di interazioni di ciascun soggetto sia confrontabile con quello degli altri non sono realistiche: anzi, la probabilità che si verifichi un incontro fra due individui presi a caso è praticamente infinitesima. Di norma, ognuno ha una serie di contatti regolari con un numero ristretto di persone (familiari, colleghi, etc), mentre ignora tutto il resto della popolazione; questo li rende particolarmente adatti ad essere rappresentati tramite una rete.
\medskip 
Andiamo ad introdurre una serie di definizioni che in seguito ci risulteranno utili. \\
\begin{definizione}[\textit{Grafo}]
Un \emph{grafo} (o una \emph{rete}) è un insieme di elementi detti vertici o \emph{nodi} che possono essere collegati fra loro da segmenti detti archi o \emph{link}.
\end{definizione}
\begin{figure}
		\begin{center}
			\includegraphics[scale=.8, keepaspectratio]{network_example}
			\caption{(a) Un grafo semplice, cioè privo di loop o link multipli. (b) Un grafo che presenta entrambi. \cite{Newman}}
			\label{fig:net_ex}
		\end{center}
	\end{figure}
	
Consideriamo una rete non orientata - cioè una rete in cui i link possono essere percorsi indistintamente in un verso e nell'altro - con $ n $ vertici, che andiamo ad etichettare da $ 1 $ a $ n $. Se indichiamo con $ \left( i, j \right) $ l'arco fra i nodi $ i $ e $ j $, allora l'intera rete può essere descritta in funzione della
\begin{definizione}[\textit{Matrice di adiacenza I}]
La matrice di adiacenza \textbf{A} relativa ad un grafo semplice è una matrice i cui elementi sono così definiti:
\[
A_{ij} \, =
\begin{cases}
1, & \text{se esiste un arco fra $ i $ e $ j $}\\ 
0, & \text{altrimenti}
\end{cases}
\]
\end{definizione}
Se dunque prendiamo, ad esempio, la rete (a) in \cref{fig:net_ex}, assumerà la seguente forma:
\begin{equation}
A =
\begin{pmatrix}
0 & 1 & 0 & 0 & 1 & 0 \\
1 & 0 & 1 & 1 & 0 & 0 \\
0 & 1 & 0 & 1 & 1 & 1 \\
0 & 1 & 1 & 0 & 0 & 0 \\
1 & 0 & 1 & 0 & 0 & 0 \\
0 & 0 & 1 & 0 & 0 & 0
\end{pmatrix}
\end{equation}
Possiamo osservare che risulta essere una matrice simmetrica con tutti $ 0 $ sulla diagonale. 
\\Qualora invece ne avessimo sotto esame una più simile alla (b) della medesima figura, dovremmo tener conto del fatto che sono presenti link multipli e loop; si stabilisce di assegnare ai primi un numero pari alla loro molteplicità e ai secondi il valore $ 2 $, così che
\begin{equation}
A =
\begin{pmatrix}
0 & 1 & 0 & 0 & 3 & 0 \\
1 & 2 & 2 & 1 & 0 & 0 \\
0 & 2 & 0 & 1 & 1 & 1 \\
0 & 1 & 1 & 0 & 0 & 0 \\
3 & 0 & 1 & 0 & 0 & 0 \\
0 & 0 & 1 & 0 & 0 & 2
\end{pmatrix}
\end{equation}
matrice che rimane, ad ogni modo, simmetrica.

\begin{wrapfigure}{l}{0.4\textwidth}
		\includegraphics[width=0.38\textwidth]{digraph_example}
		\caption{Un digrafo. \cite{Newman}}
		\label{fig:dig_ex}
\end{wrapfigure}

La questione cambia leggermente se si va a considerare una rete diretta o \emph{digrafo}.
\begin{definizione}[\textit{Digrafo}]
Un \emph{digrafo} è un tipo di grafo in cui ogni arco ha una direzione, punta cioè da un vertice ad un altro.
\end{definizione}
Ciò ci porta a rivedere quanto detto per la matrice di adiacenza: affermiamo, quindi, che
\begin{definizione}[\textit{Matrice di adiacenza II}]
La matrice di adiacenza \textbf{A} relativa ad un grafo orientato è una matrice i cui elementi sono così definiti:
\[
A_{ij} \, =
\begin{cases}
1, & \text{se esiste un arco da $ j $ a $ i $}\\ 
0, & \text{altrimenti}
\end{cases}
\]
\end{definizione}

Relativamente alla \cref{fig:dig_ex}, si avrà pertanto \\
\begin{equation}
A =
\begin{pmatrix}
0 & 0 & 0 & 1 & 0 & 0 \\
0 & 0 & 1 & 0 & 0 & 0 \\
1 & 0 & 0 & 0 & 1 & 0 \\
0 & 0 & 0 & 0 & 0 & 1 \\
0 & 0 & 0 & 1 & 0 & 0 \\
0 & 1 & 0 & 0 & 0 & 0
\end{pmatrix}
\end{equation}
\\
Ci soffermiamo poi su un particolare tipo di grafo, che ci risulterà più utile in seguito.
\begin{definizione}[\textit{Grafo bipartito}] \cite{Bondy}
Un grafo si dice \emph{bipartito} se i suoi vertici possono essere suddivisi in due sottoinsiemi X e Y tali che ogni link ha un'estremità in X ed una in Y. 
\end{definizione}
%\begin{figure}[h]
%	\begin{center}
%		\includegraphics[scale=1]{bipartite_graph}
%		\caption{Un grafo bipartito \emph{completo}, in cui ogni vertice del gruppo X è connesso a tutti quelli del gruppo Y.}
%		\label{fig:bip_graph}
%	\end{center}
%\end{figure}
In modo del tutto equivalente a quanto abbiamo già fatto, possiamo andare a definire per un grafo siffatto una matrice che lo va a descrivere.
\begin{definizione}[\textit{Matrice di incidenza}]
La matrice di incidenza \textbf{B} è una matrice $ g x n $, dove $ g $ è il numero di sottoinsiemi e $ n $ quello dei vertici che ne fanno parte, i cui elementi sono definiti come segue:
\[
B_{ij} \, =
\begin{cases}
1, & \text{se il vertice $ j $ appartiene al gruppo $ i $ }\\
0, & \text{altrimenti}
\end{cases}
\]
\end{definizione}

\begin{wrapfigure}{r}{0.7\textwidth}
	\begin{center}
		\includegraphics[width=0.4\textwidth]{projections}
	\end{center}
	\caption{
%	In figura si mette in luce il passaggio da una rappresentazione tipica di una rete bipartito ad una in cui compaiono solo i vertici. 
	\cite{Newman}}
	\label{fig:projs}
\end{wrapfigure}
	Uno dei pregi di una rete bipartita è quello di consentire in modo fluido il passaggio da una rappresentazione in cui compaiono sia vertici che gruppi ad una in cui figurano, ad esempio, soltanto i vertici; questa possibilità permette di visualizzare il formarsi di quelle che prendono il nome di cricche (\emph{cliques}).
\begin{definizione}[\textit{Cricca}] \cite{Bickle}
Una \emph{cricca} è un sottografo completo, cioè un sottografo in cui ogni vertice è connesso a tutti gli altri.
\end{definizione}
	




\chapter{Titolo del terzo capitolo}
\label{chap:cap3}
Quando parliamo di \emph{modellizzazione ad agenti} intendiamo un tipo di modellizzazione computazionale nel quale un fenomeno viene rappresentato in termine di quelli che vengono chiamati agenti e delle loro interazioni; con \emph{agente} andiamo ad indicare un individuo o un oggetto autonomo in possesso di determinate proprietà e in grado di compiere certe azioni. \\I rapporti fra le varie entità che popolano il modello avvengono localmente, cioè solo fra quelli che vengono considerati \emph{vicini}: è immediato, pertanto, pensare che i soggetti sotto esame si muovano su di una rete e che possano necessariamente avere a che fare solo con chi si trova in loro prossimità. \\ Il primo grande vantaggio derivante dall'uso di questo approccio risiede nel fatto che risulta di più semplice comprensione rispetto ad una rappresentazione di tipo matematico: dal momento che consente di proiettare negli agenti quella che è la nostra esperienza personale, modulata in termini delle interazioni con gli individui coi quali entriamo in contatto, è evidente quanto possa essere più vicino al nostro linguaggio e al nostro modo di pensare. D'altro conto, non si può negare che un'equazione, se risolubile, fornisce in modo diretto un risultato senza il bisogno di far girare un modello, che, specialmente se costituito di un gran numero di agenti, può richiedere un tempo di esecuzione talmente lungo da renderlo poco funzionale \cite{Rand}. Un ulteriore merito del quale è necessario prendere atto è la sua capacità di mettere in luce il fenomeno dell'\emph{emergenza}, cioè di tutto quell'insieme di comportamenti e proprietà che vengono fuori dall'interazione di singoli individui e che non risultano predicibili a partire dalle caratteristiche di questi ultimi; ciò è possibile perché si tratta di un approccio "across-levels", secondo il quale gli agenti non vengono considerati avulsi dall'ambiente in cui sono immersi, ma viene dato particolare rilievo alla reciproca influenza fra i due. \\ Non sarà sorprendente, a questo punto, affermare che il processo di modellizzazione sottostà ad una serie di passi che possono venire iterati più volte - dando luogo a quello che chiamiamo "modeling cycle" - in modo da rendere il modello il più efficace ed efficiente possibile:
\begin{enumerate}
\item \textit{formulazione delle domande}, per meglio mettere a fuoco il problema che si vuole analizzare;
\item \textit{costruzione delle ipotesi}, che devono inizialmente essere il più semplici possibili, per venir poi rinforzate ed arricchite più avanti nel processo;
\item \textit{scelta delle variabili di stato e dei parametri}, per iniziare a mettere per iscritto il modo in cui ci aspettiamo che il nostro modello si comporti;
\item \textit{implementazione del modello}, che consente di esplorare e valutare se le assunzioni fatte sono valide ed hanno prodotto un che di utile;
\item \textit{analisi, test e revisione}, così da apportare modifiche o migliorie;
\item \textit{comunicazione del modello} \cite{Grimm}.
\end{enumerate}
In ultimo, è interessante sottolineare, seppur brevemente, quale sia il ruolo epistemico della simulazione computazionale: è, infatti, possibile attribuirle un potere sia \emph{esplicativo}, poiché permette di comprendere come si siano verificati alcuni eventi, sia \emph{predittivo}, dal momento che rende possibile immaginare il comportamento futuro di un sistema sotto determinate circostanze, che \emph{esplorativo}, in quanto fa sì che le informazioni che intendiamo rappresentare possano essere condivise e rese note ad altri \cite{Primiero}.
\\ All'interno di questo lavoro di tesi faremo uso di Netlogo, che è sia un IDE, ovvero un ambiente di sviluppo integrato, che un linguaggio di modellazione; è stato progettato nel $ 1999 $ col fine di rendere il più semplice possibile la costruzione di modelli basati su agenti. Nella fattispecie, utilizzeremo la versione $ 6.2.0 $; per ulteriori dettagli tecnici, rimandiamo alla documentazione ufficiale \cite{Wilensky}.
% \addcontentsline{toc}{chapter}{Conclusioni}
% \chapter*{Conclusioni}
\chapter{Conclusioni}
\label{chap:conclusioni}

	
	
	

%---------------------APPENDIX---------------------------------
\appendix
\chapter{Codice sviluppato: completo e commentato}
\label{appendix:code}
Vediamo qui riportato il codice NetLogo sviluppato; tenendo conto della percezione del rischio, va a modellare la diffusione di un'epidemia su due diversi tipi di rete.
\begin{center}
	\lstinputlisting[
		language={NetLogo}, 
		firstline=1, 
		lastline=114,
		label={list:NetLogo_risk_perception}, 
		frame={none},
		basicstyle={\small\ttfamily}
	]{other/code/riskperception-turtles-with-comments.nlogo}
\end{center}

%\begin{center}
%	\begin{lstlisting}[autogobble,language={NetLogo},caption={Codice NetLogo che modella la diffusione di un'epidemia su due diversi tipi di rete tenendo conto della percezione del rischio.},label={list:NetLogo_risk_perception}]
%;; qui vengono specificate le variabili in termine delle quali si descrive lo stato degli agenti ;;
%turtles-own [state infected? alert]
%;; procedura di setup, il cui primario obiettivo è quello di ripulire il mondo da precedenti esecuzioni del modello ;;
%;: crea innanzitutto 2 agenti, li fa allontanare di 5 passi e li lega mediante un arco ;;
%;; mediante un'estrazione di un numero casuale fra 0 e 1, genera una certa quantità di infetti ;;
%;; mentre tutti gli altri sono suscettibili ;;
%to setup
%  clear-all
%  no-display
%  set-default-shape turtles "circle"
%  ;; create two turtles (nodes) and space them out
%  create-turtles 2 [
%    fd 5
%  ]
%  ask turtle 0 [ create-link-with turtle 1 ]
%  repeat num-nodes [
%    create-network
%  ]
%  ask turtles [
%    set state 0
%    set infected? false
%  ]
%  ask turtles [
%    if random-float 1 < initial-infected [
%      set state num-states
%      set infected? true
%    ]
%  ]
%  recolor
%  display
%  reset-ticks
%end
%
%;; procedura che regola il processo di diffusione dell'infezione ;;
%to go
%  if (not any? turtles with [color = red] or not any? turtles with [color = green]) [stop] ;; il programma si deve fermare se non ci sono più agenti verdi o rossi
%;; decrementiamo la variabile state, che tiene conto della durata dell'infezione, di un'unità ;;
%  ask turtles [
%    if state > 0 [
%      set state state - 1
%    ]
%  ]
%;; gli agenti contano il numero di vicini infetti e assegnano questo numero alla variabile alert ;;
%  ask turtles with [state > 0] [
%    ask out-link-neighbors with [state = 0] [
%      set alert count out-link-neighbors with [state > 0]
%    ]
%  ]
%;;
%  ask turtles with [state > 0] [
%    ask out-link-neighbors with [state = 0] [
%      if random-float 1 < infectivity * exp(-1 * risk-perception * alert) [
%        set state num-states
%        set infected? true
%      ]
%    ]
%  ]
%  recolor
%  tick
%end
%
%;; metodo grazie al quale gli agenti vengono colorati in base al loro stato di salute ;;
%to recolor
%  ask turtles [
%    ifelse state = 0 [
%      ifelse infected? [
%        set color yellow
%      ][
%        set color green
%      ]
%    ][
%      set color red
%    ]
%  ]
%end
%
%;; metodo che viene chiamato nella procedura to go ;;
%;; seleziona in modo casuale uno degli agenti, al quale viene connesso un nuovo nodo appena generato ;;
%;; il processo viene ripetuto per un numero di volte pari al numero di nodi, impostato tramite slider, così da creare l'intera rete ;;
%to create-network
%  if count turtles > num-nodes [ stop ]
%  let partner one-of turtles
%  if random-float 1 < preferential-attachment [
%    set partner one-of [ both-ends ] of one-of links
%  ]
%  create-turtles 1 [
%    move-to partner
%    fd 1
%    create-link-with partner
%  ]
%  layout
%end
%
%;; procedura con la quale si stabilisce la dimensione degli agenti basandoci sul valore dello slider display-degree ;;
%;; e si opera sull'aspetto della rete, in modo che i nodi legati fra loro non risultino né troppo vicini né troppo lontani ;;
%;; e che non si ammassino in corrispondenza dei bordi ;;
%to layout
%  ask turtles [
%    ifelse display-degree?
%      [ set size sqrt count my-links ]
%      [ set size 1 ]
%  ]
%  layout-spring turtles links 0.2 2.0 0.5
%
%  ask turtles [
%    facexy 0 0
%    fd (distancexy 0 0) / 100
%  ]
%end
%
%;; metodo che riporta il numero di agenti che non si sono mai infettati rispetto al loro numero totale ;;
%to-report fraction-susceptibles
%  report count turtles with [infected? = false] / count turtles
%end
%	\end{lstlisting}
%\end{center}
%%\chapter{Acronimi}
% nel caso volessimo spostare acronimi nel backmatter
\addcontentsline{toc}{chapter}{Acronimi}
\chapter*{Acronimi} \markboth{}{Acronimi}
\label{appendix:acro}
%	\begin{acronym}[CAGD]
%	  	\acro{USB}{Universal Serial Bus}
%	\end{acronym}
%--------------------------------------------------------------

%---------------------BACK-MATTER------------------------------
\backmatter
% The back matter may contain such things as a glossary, notes, a bibliography, and an index.

%\chapter{Acronimi}
% nel caso volessimo spostare acronimi nel backmatter
\addcontentsline{toc}{chapter}{Acronimi}
\chapter*{Acronimi} \markboth{}{Acronimi}
\label{appendix:acro}
%	\begin{acronym}[CAGD]
%	  	\acro{USB}{Universal Serial Bus}
%	\end{acronym}

% Indice analitico -> usare comando "makeindex" in fase di compilazione
%\printindex

% Elenco Immagini
\listoffigures

% Elenco Codici
\lstlistoflistings

% Elenco Tabelle
\begingroup
\let\clearpage\relax
\let\cleardoublepage\relax
\let\cleardoublepage\relax
\vspace*{8ex}
\listoftables
\endgroup 

% Bibliografia -> usare comando "bibtex" in fase di compilazione
\bibliography{4-back/bibliografia}
% \bibliographystyle{unsrt}
\bibliographystyle{babunsrt-fl}
\addcontentsline{toc}{chapter}{Bibliografia}


% Ringraziamenti
\chapter*{Ringraziamenti}
\thispagestyle{empty}

\begin{flushleft}

\textit{Desidero ringraziare tutti i docenti del corso di laurea, che, nel corso di questi anni, hanno contribuito alla mia formazione. 
\medskip
\\Un ringraziamento in particolare va al professor Franco Bagnoli, per la dedizione ed il tempo dedicatomi durante la stesura di questo lavoro di tesi.}
%
%\medskip
%
%\textit{Ringraziamento strappalacrime 2.}

\end{flushleft}

%--------------------------------------------------------------
\end{document}