\chapter{Titolo del primo capitolo}
\label{chap:cap1}
L'obiettivo dietro alla creazione di un modello matematico di una malattia infettiva è quello di arrivare a comprenderne e a descriverne il processo di trasmissione. In linea del tutto generale, possiamo andare a semplificarlo come segue:
%\begin{enumerate}
%\item \textit{a.} in primo luogo, uno o più soggetti infetti vengono introdotti in una popolazione di individui suscettibili (a rischio, cioè, di contrarre la malattia);
%\item \textit{b.} un individuo che viene infettato può inizialmente rimanere asintomatico, per poi mostrare i sintomi; può guarire, sia grazie all'assunzione di medicinali che all'azione del sistema immunitario, ed acquisire così una protezione nei confronti di una possibile reinfezione;
%\item \textit{c.} quando il bacino dei potenziali suscettibili viene sufficientemente svuotato, la diffusione inizia a rallentare fino a fermarsi; se vengono aggiunti nuovi soggetti alla popolazione, che sia a seguito di flussi migratori o di nascite, l'epidemia può persistere per un lungo periodo di tempo e diventare così endemica. \\
%\end{enumerate}

\begin{enumerate}
\item[a.] in primo luogo, uno o più soggetti infetti vengono introdotti in una popolazione di individui suscettibili (a rischio, cioè, di contrarre la malattia);
\item[b.] un individuo che viene infettato può inizialmente rimanere asintomatico, per poi mostrare i sintomi; può guarire, sia grazie all'assunzione di medicinali che all'azione del sistema immunitario, ed acquisire così una protezione nei confronti di una possibile reinfezione;
\item[c.] quando il bacino dei potenziali suscettibili viene sufficientemente svuotato, la diffusione inizia a rallentare fino a fermarsi; se vengono aggiunti nuovi soggetti alla popolazione, che sia a seguito di flussi migratori o di nascite, l'epidemia può persistere per un lungo periodo di tempo e diventare così endemica. \\
\end{enumerate}

La modellazione matematica si è rivelata di centrale importanza nel saper rispondere alle domande che possono sorgere all'alba di quella che, a tutti gli effetti, potrebbe rivelarsi una nuova epidemia (se non una pandemia), quali, ad esempio, quale possa essere il numero di persone bisognose di cure ospedaliere o quali effetti possa sortire l'imposizione di una quarantena. La sua forza sta anche nel fatto che gli approcci tradizionali, quello statistico e quello sperimentale, in questo frangente non si rivelano altrettanto utili: se da una parte diventa complicato riprodurre in laboratorio il comportamento su grande scala di una malattia infettiva, che può coinvolgere un gran numero di persone distribuite in aree geografiche spazialmente estese, dall'altra è difficile fare affidamento su di un'analisi statistica se i dati raccolti non sono completi o accurati (basti pensare alla difficoltà di reperire informazioni su soggetti asintomatici). \\ L'obiettivo della modellazione è, dunque, triplice \cite{Daley}:
\begin{enumerate}
\item come già detto, consentire una migliore comprensione dei meccanismi di trasmissione dell'infezione;
\item riuscire, di conseguenza, a predirne l'andamento futuro;
\item infine, individuare delle modalità di contenimento per tenere sotto controllo la diffusione.
\end{enumerate}
Il processo che si mette in atto consiste di una serie di passi, che vanno dalla formulazione di assunzioni sulla trasmissione della malattia, a partire dalle quali si può costruire un primo modello, alla validazione dello stesso mediante i dati raccolti. C'è, tuttavia, da tenere conto del fatto che un modello matematico non è che un'approssimazione, basata sulle ipotesi che facciamo a seguito di quanto siamo riusciti ad osservare; ciò si traduce nella necessità, imposta anche da una non adeguata conoscenza della malattia in questione, di fare ricorso a delle semplificazioni: risulta, pertanto, chiara l'esigenza di andare a confermare coi dati, qualora possibile, il modello che si è messo in piedi. \\ Ci sono generalmente tre approcci che si possono seguire:
\begin{itemize}
\item quello dei modelli statistici, ampiamente usati in epidemiologia, ma con lo svantaggio, come abbiamo già sottolineato, di necessitare di grandi campioni di dati;
\item quello dei modelli deterministici, retti dall'assunzione secondo la quale la dimensione delle popolazioni dei suscettibili e degli infetti sia una funzione continua del tempo; risultano meno affidabili se queste ultime constano di pochi individui, ma anche matematicamente maturi e meno dipendenti dai dati;
\item infine, quello dei modelli stocastici, che ben si adattano ad essere impiegati nel caso in cui si abbia a che fare con gruppi ristretti, ma che, al contempo, hanno bisogno di un gran numero di simulazioni numeriche.  
\end{itemize}
\cite{Li} 