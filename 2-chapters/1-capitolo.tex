\chapter{Modellare un'epidemia}
\label{chap:cap1}
L'obiettivo dietro alla creazione di un modello matematico di una malattia infettiva è quello di arrivare a comprenderne e a descriverne il processo di trasmissione. In linea del tutto generale, possiamo andare a semplificarlo come segue:
%\begin{enumerate}
%\item \textit{a.} in primo luogo, uno o più soggetti infetti vengono introdotti in una popolazione di individui suscettibili (a rischio, cioè, di contrarre la malattia);
%\item \textit{b.} un individuo che viene infettato può inizialmente rimanere asintomatico, per poi mostrare i sintomi; può guarire, sia grazie all'assunzione di medicinali che all'azione del sistema immunitario, ed acquisire così una protezione nei confronti di una possibile reinfezione;
%\item \textit{c.} quando il bacino dei potenziali suscettibili viene sufficientemente svuotato, la diffusione inizia a rallentare fino a fermarsi; se vengono aggiunti nuovi soggetti alla popolazione, che sia a seguito di flussi migratori o di nascite, l'epidemia può persistere per un lungo periodo di tempo e diventare così endemica. \\
%\end{enumerate}
\begin{enumerate}
\item[a.] in primo luogo, uno o più soggetti infetti vengono introdotti in una popolazione di individui suscettibili (a rischio, cioè, di contrarre la malattia);
\item[b.] un individuo che viene infettato può inizialmente rimanere asintomatico, per poi mostrare i sintomi; può guarire, sia grazie all'assunzione di medicinali che all'azione del sistema immunitario, ed acquisire così una protezione nei confronti di una possibile reinfezione;
\item[c.] quando il bacino dei potenziali suscettibili viene sufficientemente svuotato, la diffusione inizia a rallentare fino a fermarsi; se vengono aggiunti nuovi soggetti alla popolazione, che sia a seguito di flussi migratori o di nascite, l'epidemia può persistere per un lungo periodo di tempo e diventare così endemica. \\
\end{enumerate}
\section{Perch\'{e} modelliamo: obiettivi e limiti}
La modellazione matematica si è rivelata di centrale importanza nel saper rispondere alle domande che possono sorgere all'alba di quella che, a tutti gli effetti, potrebbe rivelarsi una nuova epidemia (se non una pandemia), come, ad esempio, quale possa essere il numero di persone bisognose di cure ospedaliere o quali effetti possa sortire l'imposizione di una quarantena. La sua forza sta anche nel fatto che gli approcci tradizionali, quello statistico e quello sperimentale, in questo frangente non si rivelano altrettanto utili: se da una parte diventa complicato riprodurre in laboratorio il comportamento su grande scala di una malattia infettiva, che può coinvolgere un gran numero di persone distribuite in aree geografiche spazialmente estese, dall'altra è difficile fare affidamento su di un'analisi statistica se i dati raccolti non sono completi o accurati (basti pensare alla difficoltà di reperire informazioni su soggetti asintomatici). \\ Lo scopo della modellazione si fa, dunque, triplice:
\begin{enumerate}
\item come già detto, consentire una migliore comprensione dei meccanismi di trasmissione dell'infezione;
\item riuscire, di conseguenza, a predirne l'andamento futuro;
\item infine, individuare delle modalità di contenimento per tenere sotto controllo la diffusione \cite{Daley}.
\end{enumerate}
Il processo che si mette in atto consiste di una serie di passi, che vanno dalla formulazione di assunzioni sulla trasmissione della malattia, a partire dalle quali si può costruire un primo modello, alla validazione dello stesso mediante i dati raccolti. C'è, tuttavia, da tenere conto del fatto che un modello matematico non è che un'approssimazione, basata sulle ipotesi che facciamo a seguito di quanto siamo riusciti ad osservare; ciò si traduce nella necessità, imposta anche da una non adeguata conoscenza della malattia in questione, di fare ricorso a delle semplificazioni: risulta, pertanto, chiara l'esigenza di andare a confermare coi dati, qualora possibile, il modello che si è messo in piedi. 
\section{Modelli compartimentali}
Ci sono generalmente tre approcci che si possono seguire:
\begin{itemize}
\item quello dei modelli statistici, ampiamente usati in epidemiologia, ma con lo svantaggio, come abbiamo già sottolineato, di necessitare di grandi campioni di dati;
\item quello dei modelli deterministici, 
%retti dall'assunzione secondo la quale la dimensione delle popolazioni dei suscettibili e degli infetti sia una funzione continua del tempo
che prendono in considerazione soltanto l'evoluzione temporale dei valori medi delle dimensioni delle varie popolazioni; risultano meno affidabili se queste ultime constano di pochi individui, tuttavia sono matematicamente maturi e meno dipendenti dai dati;
\item infine, quello dei modelli stocastici, che ben si adattano ad essere impiegati nel caso in cui si abbia a che fare con gruppi ristretti, ma che, al contempo, hanno bisogno di un gran numero di simulazioni numeriche \cite{Li}.  
\end{itemize}
In primo luogo, si va a suddividere la popolazione sotto esame in gruppi mutualmente esclusivi - come se fossero, per l'appunto, \emph{compartimenti} stagni - così che questi possano riflettere caratteristiche osservabili del processo di infezione. Come evidenziato in \cite{Kiss}, quello che ci si propone di fare è descrivere nel modo più formale possibile le transizioni da una classe all'altra, così che si possa tener traccia del numero di individui che vi appartengono; detto risultato può essere raggiunto in modo diverso a seconda di quale tra i già citati approcci si sta seguendo: 
%nella fattispecie, il caso deterministico tratta le dimensioni dei compartimenti come variabili e porta alla scrittura di un sistema di equazioni differenziali ordinarie che ne seguono l'evoluzione, mentre quello stocastico 
nella fattispecie, quello deterministico, che richiede la scrittura di un sistema di equazioni differenziali ordinarie ove le variabili in gioco sono le dimensioni dei vari compartimenti, consente di conoscere in modo esaustivo il comportamento di una popolazione a partire dalle sue condizioni iniziali, mentre quello stocastico si basa sull'esistenza implicita di una rete di connettività fra gli individui e, pertanto, tratta le informazioni relative all'infezione in funzione del rate di spostamento fra gli elementi di un reticolo.
\subsection{Il modello SIR}

\begin{figure}
		\begin{center}
			\includegraphics[scale=.8, keepaspectratio]{sir_model}
			\caption{Dipendenza temporale delle dimensioni relative dei compartimenti rispetto alla totalità della popolazione; $ \gamma = 1 $ e $ \beta = 1.6 $ \cite{Kiss}.}
			\label{fig:evolution}
		\end{center}
	\end{figure}
	
Sulla falsariga del modello più noto e forse storicamente più rilevante, quello delineato da Kermack e McKendrick nel 1927 \cite{Kermack}, possiamo ora introdurre il modello SIR: esso comporta la ripartizione del campione in esame in tre gruppi, ovvero
\begin{itemize}
\item S, i \emph{suscettibili}
\item I, gli \emph{infetti}
\item R, interpretabile sia come \emph{guariti} (recovered) che come \emph{rimossi} (removed)
\end{itemize}
la cui evoluzione temporale - messa in evidenza nella \cref{fig:evolution} - è ben descritta dal seguente sistema:
\begin{equation}
\begin{cases}
\frac{dS(t)}{dt} = - \beta I(t) \frac{S(t)}{N}; \\
\frac{dI(t)}{dt} = \beta I(t) \frac{S(t)}{N} - \gamma I(t); \\
\frac{dR(t)}{dt} = \gamma I(t).
\end{cases} 
\label{SIR}
\end{equation}
laddove $\beta$ e $\gamma$, entrambe costanti $> 0$, sono rispettivamente il rate d'infezione e di guarigione/rimozione, mentre $S(t) + I(t) + R(t) = N$ fornisce la dimensione della popolazione, anch'essa una quantità costante. Al sistema soprastante vengono aggiunte anche le opportune condizioni iniziali: $S(0) = S_0 > 0$, $I(0) = I_0 > 0$, $R(0) = 0$. 
\\Osserviamo che il modello non perde di senso fintanto che $S(t)$ e $I(t)$ rimangono non negativi;
% è, pertanto, sufficiente che anche solo uno dei due diventi pari a zero perché ciò accada. 
qualora anche solo uno dei due diventi pari a zero, affermiamo che il sistema si arresta: in effetti, una qualsiasi soluzione $ \left( I = 0, S + N = R \right)$ è un punto fisso per il sistema. Notiamo, inoltre, che $\frac{dS(t)}{dt} < 0$ per ogni possibile valore di t, mentre $\frac{dI(t)}{dt} > 0$ se e solo se $\frac{\beta S(t)}{\gamma} > 1$: ciò significa che, fin quando quest'ultima disuguaglianza rimane vera, $I(t)$ cresce, ma in seguito, in virtù del fatto che $S(t)$ diminuisce a prescindere da t, inverte il suo comportamento e finisce per tendere a zero. \\ Quanto appena detto ci dà il via per andare a definire una quantità fondamentale nello studio della diffusione di una malattia infettiva, il \emph{numero di riproduzione di base} $\mathcal{R}_0$ 
\begin{equation}
\mathcal{R}_0 = \frac{\beta N}{\gamma}
\end{equation}
che conta quanti individui, in media, vengono contagiati da un singolo infetto in una popolazione che altrimenti sarebbe totalmente sana. In aggiunta, è un buon indicatore dell'eventuale esplodere o meno dell'epidemia; più precisamente
\[
\text{se} \,\mathcal{R}_0
\begin{cases}
< 1, & \text{allora $I < 0$ e il processo d'infezione si arresta}\\
> 1, & \text{allora $I$ cresce e l'epidemia prende piede}
\end{cases}
\]
La transizione fra uno scenario e l'altro si ha per $\mathcal{R}_0 = 1$, punto che prende il nome di \emph{soglia epidemica}. \\ Combinando le prime due equazioni \cite{Brauer} che compaiono in \eqref{SIR}, si ottiene
\begin{equation}
\log \frac{S_0}{S_\infty} = \frac{\beta}{\gamma} \left[ N - S_\infty \right ] = \mathcal{R}_0 \left[1 - \frac{S_\infty}{N} \right ]
\end{equation}
%
che mette in luce la relazione fra il numero di riproduzione di base e la dimensione finale dell'epidemia, $\left[ N - S_\infty \right ] $.\footnote{Qui abbiamo sottinteso di star cercando delle soluzioni costanti della coppia di equazioni differenziali, $ \left( S_\infty, I_\infty \right ) $; partendo dal presupposto che, a $ t = 0 $, $ S_0 + I_0 = N $ e ricordando che $ lim_{t\to \infty} I(t) = I_\infty = 0 $, risulta che $ \lim_{t\to \infty} S(t) + I(t) = S_\infty $ \cite{Brauer}.}

\section{Modellizzazione ad agenti}
Quando parliamo di \emph{modellizzazione ad agenti} intendiamo un tipo di modellizzazione computazionale nel quale un fenomeno viene rappresentato in termine di quelli che vengono chiamati agenti e delle loro interazioni; con \emph{agente} andiamo ad indicare un individuo o un oggetto autonomo in possesso di determinate proprietà e in grado di compiere certe azioni. \\I rapporti fra le varie entità che popolano il modello avvengono localmente, cioè solo fra quelli che vengono considerati \emph{vicini}: è immediato, pertanto, pensare che i soggetti sotto esame si muovano su di una rete e che possano necessariamente avere a che fare solo con chi si trova in loro prossimità. \\ Il primo grande vantaggio derivante dall'uso di questo approccio risiede nel fatto che risulta di più semplice comprensione rispetto ad una rappresentazione di tipo matematico: dal momento che consente di proiettare negli agenti quella che è la nostra esperienza personale, modulata in termini delle interazioni con gli individui coi quali entriamo in contatto, è evidente quanto possa essere più vicino al nostro linguaggio e al nostro modo di pensare. D'altro conto, non si può negare che un'equazione, se risolubile, fornisce in modo diretto un risultato senza il bisogno di far girare un modello, che, specialmente se costituito di un gran numero di agenti, può richiedere un tempo di esecuzione talmente lungo da renderlo poco funzionale \cite{Rand}. Un ulteriore merito del quale è necessario prendere atto è la sua capacità di mettere in luce il fenomeno dell'\emph{emergenza}, cioè di tutto quell'insieme di comportamenti e proprietà che vengono fuori dall'interazione di singoli individui e che non risultano predicibili a partire dalle caratteristiche di questi ultimi; ciò è possibile perché si tratta di un approccio "across-levels", secondo il quale gli agenti non vengono considerati avulsi dall'ambiente in cui sono immersi, ma viene dato particolare rilievo alla reciproca influenza fra i due. \\ Non sarà sorprendente, a questo punto, affermare che il processo di modellizzazione sottostà ad una serie di passi che possono venire iterati più volte - dando luogo a quello che chiamiamo "modeling cycle" - in modo da rendere il modello il più efficace ed efficiente possibile:
\begin{enumerate}
\item \textit{formulazione delle domande}, per meglio mettere a fuoco il problema che si vuole analizzare;
\item \textit{costruzione delle ipotesi}, che devono inizialmente essere il più semplici possibili, per venir poi rinforzate ed arricchite più avanti nel processo;
\item \textit{scelta delle variabili di stato e dei parametri}, per iniziare a mettere per iscritto il modo in cui ci aspettiamo che il nostro modello si comporti;
\item \textit{implementazione del modello}, che consente di esplorare e valutare se le assunzioni fatte sono valide ed hanno prodotto un che di utile;
\item \textit{analisi, test e revisione}, così da apportare modifiche o migliorie;
\item \textit{comunicazione del modello} \cite{Grimm}.
\end{enumerate}
In ultimo, è interessante sottolineare, seppur brevemente, quale sia il ruolo epistemico della simulazione computazionale: è, infatti, possibile attribuirle un potere sia \emph{esplicativo}, poiché permette di comprendere come si siano verificati alcuni eventi, sia \emph{predittivo}, dal momento che rende possibile immaginare il comportamento futuro di un sistema sotto determinate circostanze, che \emph{esplorativo}, in quanto fa sì che le informazioni che intendiamo rappresentare possano essere condivise e rese note ad altri \cite{Primiero}.
\\ All'interno di questo lavoro di tesi faremo uso di Netlogo, che è sia un IDE, ovvero un ambiente di sviluppo integrato, che un linguaggio di modellazione; è stato progettato nel $ 1999 $ col fine di rendere il più semplice possibile la costruzione di modelli basati su agenti. Nella fattispecie, utilizzeremo la versione $ 6.2.0 $; per ulteriori dettagli tecnici, rimandiamo alla documentazione ufficiale \cite{Wilensky}.