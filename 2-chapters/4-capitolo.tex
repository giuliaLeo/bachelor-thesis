% \addcontentsline{toc}{chapter}{Conclusioni}
% \chapter*{Conclusioni}
\chapter{Simulazioni}
\label{chap:cap4}
Come già accennato nel primo capitolo, in questo lavoro di tesi ci siamo serviti di Netlogo al fine di costruire un modello di diffusione epidemica su rete; in particolare, abbiamo fatto uso di BehaviorSpace, un software, messo a disposizione da Netlogo stesso, che consente di eseguire più volte un esperimento andandone a variare alcuni parametri caratteristici \cite{Wilensky2}.
	
	
	