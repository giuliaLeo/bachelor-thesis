\addcontentsline{toc}{chapter}{Introduzione}
\chapter*{Introduzione} \markboth{}{Introduzione}
% l'uso di '*' rende il capitolo non numerato ed indicizzato nel toc
% lo aggiungiamo noi al toc tramita la prima istruzione
% con \markboth{<left output>{<right output>} evitiamo che nell'header compaia il titolo del capitolo precedente

%\chapter{Introduzione}
\label{chap:intro}
Il lavoro di tesi è suddiviso in quattro capitoli, il cui contenuto viene qui esposto brevemente:
\begin{itemize}
\item \textit{Capitolo 1}: si descrive cosa significhi modellare un'epidemia, con particolare attenzione a quali siano gli obiettivi che si intendono raggiungere e, al contempo, i limiti che si incontrano nel perseguire questo intento. Dopo aver fatto cenno all'approccio compartimentale, si esplora il modello SIR e si mette in rilievo l'importanza del numero di riproduzione di base $\mathcal{R}_0$; infine, si dedica spazio alla modellizzazione ad agenti, che annovera fra i suoi pregi la capacità di mostrare l'emergere di tutta una serie di comportamenti collettivi che altrimenti non potrebbe venire predetti.
\item \textit{Capitolo 2}: si introducono una serie di concetti teorici legati ai grafi, concetti che saranno utili per meglio descrivere l'andamento di un'epidemia su rete.
\item \textit{Capitolo 3}: si tratta della percezione del rischio e della sua influenza sull'andamento della diffusione dell'infezione; questo elemento viene matematicamente formalizzato andando a modificare la probabilità netta di contrarre la malattia mediante l'aggiunta di due fattori, il primo basato sulla frazione di contatti infetti e l'altro relativo alle misure precauzionali che possono essere messe in atto. Si esplora l'andamento dell'epidemia su due diversi tipi di rete (poissoniana e scale-free) e se ne riportano i risultati sperimentali ottenuti.
\item \textit{Capitolo 4}: facendo riferimento ai concetti introdotti e ai risultati evidenziati nei capitoli precedenti, si spiega come sia stato costruito il modello NetLogo del quale ci si è serviti per condurre una serie di simulazioni; se ne sono poi messi in luce gli esiti e il loro confermare quanto precedentemente ottenuto.
\end{itemize}
