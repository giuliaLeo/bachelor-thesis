\chapter{Titolo del terzo capitolo}
\label{chap:cap3}
Quando parliamo di \emph{modellizzazione ad agenti} intendiamo un tipo di modellizzazione computazionale nel quale un fenomeno viene rappresentato in termine di quelli che vengono chiamati agenti e delle loro interazioni; con \emph{agente} andiamo ad indicare un individuo o un oggetto autonomo in possesso di determinate proprietà e in grado di compiere certe azioni. \\I rapporti fra le varie entità che popolano il modello avvengono localmente, cioè solo fra quelli che vengono considerati \emph{vicini}: è immediato, pertanto, pensare che i soggetti sotto esame si muovano su di una rete e che possano necessariamente avere a che fare solo con chi si trova in loro prossimità. \\ Il primo grande vantaggio derivante dall'uso di questo approccio risiede nel fatto che risulta di più semplice comprensione rispetto ad una rappresentazione di tipo matematico: dal momento che consente di proiettare negli agenti quella che è la nostra esperienza personale, modulata in termini delle interazioni con gli individui coi quali entriamo in contatto, è evidente quanto possa essere più vicino al nostro linguaggio e al nostro modo di pensare \cite{Rand}. Un ulteriore merito del quale prendere atto è la sua capacità di mettere in luce il fenomeno dell'\emph{emergenza}, cioè di tutto quell'insieme di comportamenti e proprietà che vengono fuori dall'interazione di singoli individui e che non risultano predicibili a partire dalle caratteristiche di questi ultimi; ciò è possibile perché si tratta di un approccio "across-levels", secondo il quale gli agenti non vengono considerati avulsi dall'ambiente in cui sono immersi, ma viene dato particolare rilievo alla reciproca influenza fra i due \cite{Grimm}.