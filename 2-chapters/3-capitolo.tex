\chapter{Titolo del terzo capitolo}
\label{chap:cap3}
Quando parliamo di \emph{modellizzazione ad agenti} intendiamo un tipo di modellizzazione computazionale nel quale un fenomeno viene rappresentato in termine di quelli che vengono chiamati agenti e delle loro interazioni; con \emph{agente} andiamo ad indicare un individuo o un oggetto autonomo in possesso di determinate proprietà e in grado di compiere certe azioni. \\I rapporti fra le varie entità che popolano il modello avvengono localmente, cioè solo fra quelli che vengono considerati \emph{vicini}: è immediato, pertanto, pensare che i soggetti sotto esame si muovano su di una rete e che possano necessariamente avere a che fare solo con chi si trova in loro prossimità. \\ Il primo grande vantaggio derivante dall'uso di questo approccio risiede nel fatto che risulta di più semplice comprensione rispetto ad una rappresentazione di tipo matematico: dal momento che consente di proiettare negli agenti quella che è la nostra esperienza personale, modulata in termini delle interazioni con gli individui coi quali entriamo in contatto, è evidente quanto possa essere più vicino al nostro linguaggio e al nostro modo di pensare. D'altro conto, non si può negare che un'equazione, se risolubile, fornisce in modo diretto un risultato senza il bisogno di far girare un modello, che, specialmente se costituito di un gran numero di agenti, può richiedere un tempo di esecuzione talmente lungo da renderlo poco funzionale \cite{Rand}. Un ulteriore merito del quale è necessario prendere atto è la sua capacità di mettere in luce il fenomeno dell'\emph{emergenza}, cioè di tutto quell'insieme di comportamenti e proprietà che vengono fuori dall'interazione di singoli individui e che non risultano predicibili a partire dalle caratteristiche di questi ultimi; ciò è possibile perché si tratta di un approccio "across-levels", secondo il quale gli agenti non vengono considerati avulsi dall'ambiente in cui sono immersi, ma viene dato particolare rilievo alla reciproca influenza fra i due. \\ Non sarà sorprendente, a questo punto, affermare che il processo di modellizzazione sottostà ad una serie di passi che possono venire iterati più volte - dando luogo a quello che chiamiamo "modeling cycle" - in modo da rendere il modello il più efficace ed efficiente possibile:
\begin{enumerate}
\item \textit{formulazione delle domande}, per meglio mettere a fuoco il problema che si vuole analizzare;
\item \textit{costruzione delle ipotesi}, che devono inizialmente essere il più semplici possibili, per venir poi rinforzate ed arricchite più avanti nel processo;
\item \textit{scelta delle variabili di stato e dei parametri}, per iniziare a mettere per iscritto il modo in cui ci aspettiamo che il nostro modello si comporti;
\item \textit{implementazione del modello}, che consente di esplorare e valutare se le assunzioni fatte sono valide ed hanno prodotto un che di utile;
\item \textit{analisi, test e revisione}, così da apportare modifiche o migliorie;
\item \textit{comunicazione del modello} \cite{Grimm}.
\end{enumerate}
In ultimo, è interessante sottolineare, seppur brevemente, quale sia il ruolo epistemico della simulazione computazionale: è, infatti, possibile attribuirle un potere sia \emph{esplicativo}, poiché permette di comprendere come si siano verificati alcuni eventi, sia \emph{predittivo}, dal momento che rende possibile immaginare il comportamento futuro di un sistema sotto determinate circostanze, che \emph{esplorativo}, in quanto fa sì che le informazioni che intendiamo rappresentare possano essere condivise e rese note ad altri \cite{Primiero}.
\\ All'interno di questo lavoro di tesi faremo uso di Netlogo, che è sia un IDE, ovvero un ambiente di sviluppo integrato, che un linguaggio di modellazione; è stato progettato nel $ 1999 $ col fine di rendere il più semplice possibile la costruzione di modelli basati su agenti. Nella fattispecie, utilizzeremo la versione $ 6.2.0 $; per ulteriori dettagli tecnici, rimandiamo alla documentazione ufficiale \cite{Wilensky}.