% \addcontentsline{toc}{chapter}{Conclusioni}
% \chapter*{Conclusioni}
\chapter{Conclusioni}
\label{chap:conclusioni}
Al giorno d'oggi, riuscire a modellare un'epidemia è un processo che assume un'importanza sempre più centrale: un motivo fra tanti risiede nel fatto che, una volta che se ne sono appresi i meccanismi di trasmissione, rende possibile l'elaborazione di una serie di strategie volte ad ostacolarla. A questo proposito, è utile andare a suddividere la popolazione in compartimenti, ciascuno descrittivo dello stato di salute degli individui che ne fanno parte; così facendo, il nostro problema può essere interpretato in chiave della probabilità di transizione fra un gruppo e l'altro. La nostra rappresentazione si arricchisce di un ulteriore tassello nel momento in cui si descrivono i contatti fra soggetti come gli archi di una rete i cui nodi sono le persone stesse.\\
In questo lavoro di tesi ci siamo concentrati sull'influenza che la percezione del rischio ha nella diffusione di una malattia infettiva; per far ciò, abbiamo assunto che la probabilità di venire contagiati fosse modulata da altri due fattori, la frazione di contatti infetti e l'insieme di misure precauzionali assunte, che, complessivamente, la diminuiscono. Siamo partiti da un modello epidemico sviluppato su rete e, grazie all'utilizzo di BehaviorSpace, siamo riusciti a effettuare un certo numero di simulazioni: questo ci ha permesso di ripetere più volte l'esperimento e di indagare i diversi scenari di trasmissione che si sono originati dalle medesime condizioni iniziali. \\
Possibili sviluppi futuri potrebbero tentare di fornire una risposta ad un interrogativo che ci siamo posti nel capitolo $ 4 $, ovvero se, nel caso di una rete poissoniana, esista un valore di soglia per $ J $ oltre il quale la frazione di individui che non si sono mai ammalati non varia in modo significativo; come suggerito, si potrebbe ripetere l'esperimento facendo variare $ J $ su di un intervallo più ampio. Di altrettanto interesse potrebbe essere riprendere il discorso sulle reti scale-free, andando, ad esempio, a differenziare il tipo di misure di sicurezza adottate in base al numero di contatti di ciascun individuo: intervenendo su soggetti che ne incontrano molti altri, cioè sui cosiddetti \emph{hub}, ci aspettiamo di ridurre in modo sostanziale le occasioni in cui può avvenire la trasmissione e, pertanto, ostacolare la riproduzione del virus.
	
	
	