% \addcontentsline{toc}{chapter}{Conclusioni}
% \chapter*{Conclusioni}
\chapter{Conclusioni}
\label{chap:conclusioni}
Al giorno d'oggi, quello della modellazione epidemica è un processo che assume un'importanza sempre più centrale: un motivo fra tanti risiede nel fatto che, una volta che se ne sono appresi i meccanismi di trasmissione, rende possibile l'elaborazione di una serie di strategie volte ad ostacolarla. A questo proposito, è utile andare a suddividere la popolazione in compartimenti, ciascuno descrittivo dello stato di salute degli individui che ne fanno parte; così facendo, il nostro problema può essere interpretato in chiave della probabilità di transizione fra un gruppo e l'altro. La nostra rappresentazione si arricchisce di un ulteriore tassello nel momento in cui si descrivono i contatti fra soggetti come gli archi di una rete i cui nodi sono le persone stesse. 
\medskip
\\
%In questo lavoro di tesi ci siamo concentrati sull'influenza che la percezione del rischio ha nella diffusione di una malattia infettiva; per far ciò, ci siamo basati sulle assunzioni abbiamo assunto che la probabilità di venire contagiati fosse modulata da altri due fattori, la frazione di contatti infetti e l'insieme di misure precauzionali assunte, che, complessivamente, hanno come effetto quello di diminuirla. Siamo partiti da un modello epidemico sviluppato su rete e, grazie all'utilizzo di BehaviorSpace, siamo riusciti a effettuare un certo numero di simulazioni: questo ci ha permesso di ripetere più volte l'esperimento e di indagare i diversi scenari di trasmissione che si sono originati dalle medesime condizioni iniziali. 
In questo lavoro di tesi ci siamo concentrati sull'influenza che la percezione del rischio ha nella diffusione di una malattia infettiva. Per far ciò ci siamo basati su un modello epidemico sviluppato su rete da Bagnoli \textit{et al.} in \cite{Bagnoli2014} e \cite{Bagnoli2007}; in particolar modo, abbiamo sfruttato l'assunto secondo cui la probabilità di venire contagiati venga modulata da altri due fattori, la frazione di contatti infetti e l'insieme di misure precauzionali assunte (rappresentato dal parametro $ J $), che, complessivamente, hanno come effetto quello di diminuirla. Abbiamo importato tale modello in un contesto ad agenti, al fine di simulare, a partire dalle medesime condizioni iniziali, diversi scenari di trasmissione e di indagare come la differente percezione del rischio vada ad impattare sul numero di soggetti che non sono mai stati esposti al contagio. 
\medskip
\\
%
%\\Nel caso di una rete aleatoria, quello che è emerso è che basta cambiare anche di poco il valore di $ J $ per ottenere frazioni significativamente diverse degli agenti rimasti sani per tutta la durata delle simulazioni (basti confrontare l'andamento per $ J = 0.10 $ e per $ J = 0.15 $); in aggiunta, si noti in \cref{fig:sim_poisson} che per $ J \geq 0.30 $ le curve ottenute si fanno praticamente indistinguibili, il che potrebbe far ipotizzare che esista un valore di soglia per $ J $ oltre il quale non si registrano grosse differenze nel numero di individui che non hanno contratto l'infezione \footnote{Per andare a verificare questa supposizione, dovremmo ripetere l'intero esperimento facendo variare $ J $ in un range più ampio.}. \\Come ci aspettavamo, non avviene lo stesso nel caso scale free; difatti, se si tiene sott'occhio la \cref{fig:sim_scalefree}, si può osservare che non si verificano miglioramenti apprezzabili se non a partire da $ J = 0.25 $. Questo comportamento è strettamente legato alla struttura della rete: la presenza di \emph{hub}, cioè di nodi molto connessi, ha reso più complicata l'estinzione dell'epidemia proprio perché questi, in potenza, contagiano un numero di vicini ben più grande rispetto ad un vertice più isolato. Del resto, sarebbe sufficiente osservare anche solo una singola simulazione del modello NetLogo per convincersene: quand'anche la percezione del rischio fosse relativamente alta, se un hub contrae l'infezione è altamente probabile che tutti i nodi che gli sono collegati facciano la stessa fine, mentre è piuttosto verosimile che agenti nella periferia della rete non entrino mai in contatto con la malattia. Si conferma, perciò, uno dei risultati messi in evidenza in \cite{Bagnoli2007}, ovvero la necessità che gli individui con un elevato numero di contatti adottino misure di precauzione più stringenti rispetto al resto della popolazione al fine di porre un ulteriore freno al diffondersi dell'infezione.
A partire dagli esperimenti condotti, è emerso che in una rete poissoniana, in cui i link sono distribuiti in modo totalmente aleatorio, è sufficiente che il valore di $ J $ venga incrementato anche di poco per notare un rallentamento dell'epidemia e, di conseguenza, un miglioramento nel numero dei soggetti rimasti suscettibili; al contrario, nel caso di una rete scale-free, che presenta nodi più connessi rispetto ad altri, è stata necessaria l'imposizione di un livello di allerta maggiore per riuscire ad ottenere un risultato simile allo scenario precedente.\\
Possibili sviluppi futuri potrebbero tentare di fornire una risposta ad un interrogativo che ci siamo posti nel capitolo \ref{chap:cap4}, ovvero se, nel primo dei due casi presi in esame, esista un valore di soglia per $ J $ oltre il quale la frazione di individui che non si sono mai ammalati non varia in modo significativo; come suggerito, si potrebbe ripetere l'esperimento facendo variare $ J $ su di un intervallo più ampio, il che consentirebbe di valutare se la tendenza individuata sia effettivamente presente. Di altrettanto interesse potrebbe essere riprendere il discorso sulle reti scale-free, andando, ad esempio, a differenziare il tipo di misure di sicurezza adottate in base al numero di contatti di ciascun individuo: intervenendo su soggetti che ne incontrano molti altri, cioè sui cosiddetti \emph{hub}, ci aspettiamo di ridurre in modo sostanziale le occasioni in cui può avvenire la trasmissione e, pertanto, ostacolare la riproduzione del virus.
	
	
	